% !TeX encoding = UTF-8
% !TeX spellcheck = pt_BR
\documentclass[
	% -- opções da classe memoir --
	article,			% indica que é um artigo acadêmico
	11pt,				% tamanho da fonte
	oneside,			% para impressão apenas no recto. Oposto a twoside
	a4paper,			% tamanho do papel.
	% -- opções da classe hyperrref --
	hidelinks,
	% -- opções da classe abntex2 --
	%chapter=TITLE,		% títulos de capítulos convertidos em letras maiúsculas
	%section=TITLE,		% títulos de seções convertidos em letras maiúsculas
	%subsection=TITLE,	% títulos de subseções convertidos em letras maiúsculas
	%subsubsection=TITLE % títulos de subsubseções convertidos em letras maiúsculas
	% -- opções do pacote babel --
	english,			% idioma adicional para hifenização
	brazil,				% o último idioma é o principal do documento
	sumario=abnt-6027-2012
	]{abntex2}
\usepackage{lmodern}			% Usa a fonte Latin Modern			
\usepackage[T1]{fontenc}		% Selecao de codigos de fonte.
\usepackage[utf8]{inputenc}		% Codificacao do documento (conversão automática dos acentos)
\usepackage{lastpage}			% Usado pela Ficha catalográfica
\usepackage{indentfirst}		% Indenta o primeiro parágrafo de cada seção.
\usepackage{color}				% Controle das cores
\usepackage{graphicx}			% Inclusão de gráficos
\usepackage{microtype}			% para melhorias de justificação
\usepackage[brazilian,hyperpageref]{backref}	% Paginas com as citações na bibl
\usepackage[alf]{abntex2cite}	% Citações padrão ABNT

\usepackage{amsmath}			%\
\usepackage{amsfonts}			% Pacotes para expressões matemáticas
\usepackage{amssymb}			%/

%\usepackage{siunitx}			% Para unidades
\usepackage{subfig}				% Para usar \subfloat

%\sisetup{output-decimal-marker = {,}, separate-uncertainty = true}

%\numberwithin{table}{section}
%\numberwithin{figure}{section}
%\numberwithin{equation}{section} % Colocar o número da seção nas referências

% título do relatório
\title{Titulo}

% nomes dos componentes  - está um ao lado do outro
\author{
	Mateus de Carvalho Coelho \\ R.A. 156675 %\and ...proximo nome ra
}

\begin{document}

\maketitle

\tableofcontents*


\section{Objetivos}

% exemplo img
%\begin{figure}
%	\center	
%	\includegraphics[width=.5\textwidth]{objetos1.png}
%	\caption{Imagem a ser trabalhada \textit{objetos1.png}}
%	\label{fig:obj1}
%\end{figure}


\section{Esteganografia}

% exemplo varias imgs
%\begin{figure}
%	\centering
%	\subfloat[Original]{
%		\includegraphics[width=.35\textwidth]{monalisa.png}
%	}\hfill
%	\subfloat[Modificada]{
%		\includegraphics[width=.35\textwidth]{monalisa_mod.png}
%	}\\
%	\subfloat[Original]{
%		\includegraphics[width=.35\textwidth]{watch.png}
%	}\hfill
%	\subfloat[Modificada]{
%		\includegraphics[width=.35\textwidth]{watch_mod.png}
%	}\\
%	\subfloat[Original]{
%		\includegraphics[width=.35\textwidth]{peppers.png}
%	}\hfill
%	\subfloat[Modificada]{
%		\includegraphics[width=.35\textwidth]{peppers_mod.png}
%	}\\
%	\subfloat[Original]{
%		\includegraphics[width=.35\textwidth]{baboon.png}
%	}\hfill
%	\subfloat[Modificada]{
%		\includegraphics[width=.35\textwidth]{baboon_mod.png}
%	}\\
%	\caption{Comparação entre imagem original e modificada pela alteração de bits menos significativos dos pixels.}
%	\label{fig:comp}
%\end{figure}


\section{Aspectos técnicos}



\end{document}